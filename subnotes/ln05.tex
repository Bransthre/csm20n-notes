\chapter{Mathematical Definitions for Functions}

\section{Assignment and Declaration of Function}
In the realm of mathematics, we seek for a unified approach to formally define mathematical objects like functions.
Mathematical objects are objects with mathematical properties, or, ``things'' that exist in mathematics.
Previously, we discussed a few way to define and denote a function. In this section, let us discuss these idea more.

As spoken before, a function is fundamentally a relationship between sets.
A function denoted as $f: X \rightarrow Y$ has a name $f$, domain $X$, and codomain $Y$; furthermore, it matches each element of $X$ to one and only one element in $Y$.
And once again, this has two implications:
\begin{enumerate}
    \item Each element $x$ of $X$ must have a corresponding value in the codomain of function $Y$.
    \item Each element $x$ of $X$ cannot have multiple corresponding value in the codomain of function $Y$.
\end{enumerate}
Previously, in Algebra II, you may have learned such a thing called ``Vertical Line Test'', which asserts that the graph of a function on a 2-dimensional space, with the horizontal axis hosting the domain and vertical axis the codomain, must have no more than one intersection than any vertical line.
In the sense of usual single-variate functions $y = f(x)$, it is that one value of $x$ can only have one value of corresponding $y$. For example, $f(x) = x^2$ is a function that only has one intersection with any vertical line.

To declare the relationship that a function hosts between its domain and codomain: we define a function with the following prototype:
\[
    \forall x \in X, f(x) = {\rm expression\ in\ x}
\]
This definition of the function is also said to be \textbf{declarative}, because it declares properties of the function without explaining the precise algorithms (or computer instructions) of doing so.

\section{Different Expressions of A Function: Graph, Table, Procedure}
Functions may also be expressed in several different ways.

\subsection{Graph}
For a function $f: X \rightarrow Y$, its relationship provides us as many elements in $X$ as the number of corresponding values across domain and codomain: $(x, y)$, or in other words $(x, f(x))$.
Meanwhile, we may notice that the Cartesian product $X \times Y$ hosts the set of all possible pairings between elements of $X$ and elements of $Y$:
\[
    X \times Y = \{(x, y) | x \in X, y \in Y\}
\]
And the function $f$ instead hosts such set of tuples:
\[
    \{(x, f(x)) | x \in X\} \subset X \times Y
\]
We call this above set of tuples containing the function's mapping the \textbf{graph} of $f$, or, ${\rm graph}(f)$.
Therefore, a function can also be defined as a subset of the Cartesian product of its domain and codomain.

\subsection{Table}
We may also express a function in a table that maps an input value to an output value.
This is perhaps the most primordial, non-mathematical method of exhibiting a function.
For example, for a function $f: \mathbb{R} \rightarrow \text{Set of student name}$ where the input value is a Student ID, we may write such function in a table as follows:
\begin{center}
    \begin{tabular}{c|c}
        Domain: Student ID & Codomain: Student Names \\
        \hline
        3036594561 & John Doe \\
        3036594562 & Jonathon Doe \\
        $\vdots$ & $\vdots$ \\
        3036595000 & Joe Dohn
    \end{tabular}
\end{center}

\subsection{Procedure}
A procedure is an algorithm that notes how a function can be computed using algorithmic notations.
For example, to compute the factorial function, we may use the following instruction:
\begin{algorithm}
    \caption{A set of instructions for computing factorials}
    \begin{algorithmic}
        \Require $n \geq 0$
        \Ensure $y = x!$
        \State $y \gets 1$
        \State $N \gets n$
        \While{$N > 0$} \\
            $y \gets y \times N$ \\
            $N \gets N - 1$
        \EndWhile
    \end{algorithmic}
\end{algorithm}
Such set of instruction also defines our function: instead of providing the mathematical abstractions,
\[
    \forall x \in \mathbb{N}_0, f(x) = x!
\]
it provides the precise instructions required for anyone (even a computer) to compute the function.
Using procedures to define functions is otherwise known as an imperative assignment.

\section{A Review of Composition}
We can combine functions to create new functions. This mechanism is known as composition.
Without assuming anything about the correspondence of different functions' domains and codomains, for two functions $f_1$ and $f_2$, we may write the composition of these functions as:
\[
    (f_2 \circ f_1)(x) = f_2(f_1(x))
\]
such that $f_2 \circ f_1$ becomes a function that accepts any element $x$ in the domain of $f_1$ and computes the above value.

Note that, the fundamental requirement for such composition to be held is that the range of $f_1$ must eb a subset of the domain of $f_2$, such that the output of the first function $f_1$ is always within the domain of the second function $f_2$.
So, to end with a more precise definition of composition:
\begin{ln-define}{Composition of Function}{}
    Assume $f_1: X \rightarrow Y$ and $f_2: X' \rightarrow Y'$, then a composition $f_3 = f_2 \circ f_1$ would be defined as a function $f_3: X \rightarrow Y'$ and declared as:
    \[
        \forall x \in X, f_3(x) = f_2(f_1(x))
    \]
\end{ln-define}
