\chapter{Set!}

\section{Fundamental Properties of Sets}
As you may have guessed, in set theory, we mainly discuss a mathematical object called ``set``.
\begin{ln-define}{Set}{}
    A set is a collection of objects. \\
    The mathematical objects that are members of a set are called elements. \\
    Sets are normally expressed in the format of:
    \[\{\text{element 1}, \text{element 2}, \dots, \text{element n}\}\]
\end{ln-define}
The symbol representing membership in a set works as follows:
\begin{ln-symbol}{Membership Symbol}{}
    If a mathematical object $x$ is a member of a set $A$, then we may mathematically express that $x \in A$. \\
    Otherwise, for a mathematical object $y$ that does not belong to $A$, $y \notin A$.
\end{ln-symbol}
There are several more properties to sets:
\begin{bindenum}
    \item The size of a set is defined as the number of elements in it. This property of set is called \textbf{cardinality}. The cardinality of a set $A$ can be mathematically denoted as $|A|$.
    \item The equality of a set is held when two sets have exactly same elements; order and repetition does not matter, albeit in computers, many implementations of sets already do not allow repetition. We call a set whose elements can be repeated a ``multiset''.
    \item Sets whose cardinality are 0, also known as empty sets, are denoted as $\{\}$ as well as $\emptyset$. \\
\end{bindenum}
To denote a set of mathematical objects that all belong to some other set but suffices some property, we may utilize a notation called \textbf{set-builder notation}:
\begin{ln-symbol}{Set-Builder Notation}{}
    A set $A$ whose members all follow expression $exp$, and the terms of $exp$ all fit some list of conditions can be written as:
    \[\{exp : \text{condition 1}, \dots, \text{condition n}\}\]
    \tcblower
    The colon above can be replaced by a vertical bar. Since we will be using vertical bars for other purposes in later chapters, a colon is perhaps the least abusive notation to play with. \\
    But, to conform the formatting of official lecture notes, let us continue with vertical bars. \\
    For example, the list of all rational numbers can be written as:
    \[\Q = \bigg\{ \frac{p}{q} \bigg| p,q \in \R,\ q \neq 0 \bigg\}\]
\end{ln-symbol}

Some useful sets have names for the ease of references:
\begin{center}
    \begin{tabular}{c||c|c}
        Notation & Name & Example Contents \\
        \hline
        $\mathbb{N}$ & Natural Numbers & $\{1, 2, \dots\}$ \\
        $\mathbb{N}_0$ & Non-negative Integers & $\{0, 1, 2, \dots\}$ \\
        $\mathbb{Z}$ & Integers & $\{\dots, -2, -1, 0, 1, 2, \dots\}$ \\
        $\mathbb{R}$ & Real Numbers & $(-\infty, \infty)$ \\
        $\mathbb{C}$ & Complex Numbers & $\{x + {\rm i} y | x, y \in \mathbb{R}\}$
    \end{tabular}
\end{center}

\section{Relationships between Sets}
We often compare sets in terms of their sizes and elements they contain. Out of these standards for comparison, we can define the relationship between a set and another larger set that contains all elements of the former as follows:
\begin{ln-define}{Subset}{}
    If every element of a set $A$ is also in a set $B$, then $A$ is a subset of $B$. \\
    Mathematically, we write it as $A \subseteq B$. \\
    Or, stating the equivalent in an opposite direction, $B \supseteq A$, and this would state $B$ as a superset of $A$.
\end{ln-define}
And a stricter similar relationship follows:
\begin{ln-define}{Strict Subset}{}
    If $A \subseteq B$ but $A$ excludes at least an element of $B$, then we say that $A$ is a strict subset (proper subset) of $B$. \\
    Mathematically denoted, $A \subset B$.
\end{ln-define}
Utilizing the definitions above, we may form some observations:
\begin{bindenum}
    \item The empty set is a proper subset of any nonempty set.
    \item The empty set is a subset for any set, including itself's.
    \item While every set is not a proper subset of itself, every set is a subset of itself.
\end{bindenum}
While we compare sets based on their members, we can also ``add``, create sets based on the members of two sets. There are two ways of doing so, being \textbf{intersection} and \textbf{union}.
\begin{ln-define}{Intersection}{}
    The intersection of set $A$ and $B$, written as $A \cap B$, is the set containing all elements that are both in A and B. \\
    In set builder notation:
    \[A \cap B = \{x | x \in A \land x \in B\}\]
\end{ln-define}
\begin{ln-define}{Union}{}
    The union of set $A$ and $B$, written as $A \cup B$, is the set of all elements contained in either $A$ or $B$. \\
    In set builder notation, it pronounces very similarly with the notation for intersections:
    \[A \cup B = \{x | x \in A \lor x \in B\}\]
\end{ln-define}
Regarding the way empty sets work in the above arithmetic, for an arbitrary set $A$:
\begin{bindenum}
    \item $A \cap \emptyset = \emptyset$
    \item $A \cup \emptyset = A$
\end{bindenum}
At last, let us regard the notion of \textbf{``complements'', the difference between sets}.
\begin{ln-define}{Complement}{}
    Imagine the arithmetic of sets to work solely upon their members, and let the difference of sets be defined such that:
    \[A - B = A \backslash B = \{x \in A | x \notin B\}\]
    This set $A \ B$ is called the set difference between A and B, or alternatively, the relative complement of B in A.
\end{ln-define}
This also indicates that the ``subtraction arithmetic'' for sets is not commutative. Using empty sets as examples:
\begin{bindenum}
    \item $A\ \backslash\ \emptyset = A$
    \item $\emptyset\ \backslash\ A = \emptyset$
\end{bindenum}
And the last arithmetic is the \textbf{``multiplication of sets'': Cartesian Products}.
\begin{ln-define}{Cartesian Products}{}
    The Cartesian Product (cross product) of sets $A$ and $B$ is defined such that:
    \[A \times B = \{(a, b)\ |\ a \in \R, b \in \R\}\]
\end{ln-define}
And last but not least, we have mathematical operations that generate a set of sets (nested) based on other sets:
\begin{ln-define}{Power Set}{}
    The power set of $S$ can be written as:
    \[\wp (S) = \{P | P \subseteq S\}\]
    as one of its various denotations, the one on lecture note using a Weierstrass P. \\
    The power set of a set is the set of all of its possible subsets.
\end{ln-define}
In addition, if the cardinality of $S$ in the above definition is $|S| = k$, then we can state that $|\mathbb{P} (S)| = 2^k$. This will be an explored notion as we discuss a discrete math topic ``Counting'' in near future.
