\chapter{Fourier Expansion: Infinite Terms}

\section{Fourier Series}
\subsection{Purpose and Definition}
\begin{ln-define}{Fourier Series}{}
    A Fourier Series is an approximation of an arbitrary signal, in the mathematical prototype:
    \[
        x(t) = A_0 + \sum_{k=1}^\infty A_k \cos(k \omega_0 t + \phi_k)
    \]
    where, as introduced before, the coefficients $A_k$ resemble the amplitude of the $k$-th term in Fourier Series, while $\phi_k$ represent its phase.
    Particularly, the value $\omega_0$ represents a fundamental frequency which all signals share.
    Mathematically, we may interpret $\omega_0$ as the greatest common divisor of all signals in the Fourier Series.
\end{ln-define}
Following the above interpretation, signal with fundamentaly frequency $\omega_0$ would therefore have a period $p = \frac{2\pi}{\omega}$.
In addition, the constant term $A_0$ is sometimes called the DC (direct current) term, and the terms for which $k \geq 2$ are called harmonics.
You will have a better intuition of this from EECS 16B when you understand the difference between a direct current and an alternating current.

Notably, not all Fourier Series are supposed to be infinite. A finite fourier series approximation follows the mathematical prototype of:
\[
    \tilde{x}(t) = A_0 + \sum_{k = 1}^K A_k \cos(k \omega_0 t + \phi_k)
\]
And we would see a reprise of an interesting statement we made before: that, signals are a weighted signals of some other sinusoidal waves.
To express a signal as a weighted sum of other signals, we visualize a signal via the coefficient of its weighted sum:
\[
    \tilde{x}(t) = \begin{bmatrix} A_0 \\ A_1 \\ \vdots \\ A_K \end{bmatrix} \begin{bmatrix} 1 & \cos(\omega_0 t + \phi_1) & \cdots & \cos(\omega_K t + \phi_K) \end{bmatrix}
\]
as if the signal $\tilde{x}(t)$ lies in a coordinate system based on its coefficients $A_k$. We call this representation a \textbf{frequency domain} representation for its association with the amplitude-frequency weighted sums as shown above.

For any periodic signal $x: \mathbb{R} \rightarrow \mathbb{R}$ with period $p$, the uniqueness of its Fourier Series is guaranteed.

\subsection{Fourier Series Approximation and Convergence}
Now that we have introduced such a technique, we would naturally question: does Fourier Series always work?
The answer is negative.
A termniology, \textbf{Gibb's phenomenon}, specifically describes the situation where a Fourier Series fails to perfectly capture a specific signal $x(t)$.
The mathematical depiction of such situation would be that as the amount of terms ($k$) approaches infinity in our Fourier Series expansion $\tilde{x}(t)$ of the signal $x(t)$, the difference between $x(t)$ and $\tilde{x}(t)$ doesn't converge to $0$.

For a Fourier Series expansion of some periodic signal $x$:
\[
    \forall t \in \mathbb{R}, x_N(t) = A_o + \sum_{k = 1}^N A_k \cos(k \omega_0 t + \phi_k)
\]
where the validity of such representation holds when
\[
    x(t) = \lim_{N \rightarrow \infty} x_N(t)
\]

We will discuss more advanced techniques of finding a Fourier Series approximation (a.k.a. solving for the coefficients $A_k$ and phases $\phi_k$), but let us also discuss the notion of whether a Fourier Series is valid (can approximate the target signal accurately).
We primarily work with two criterions of validity: uniform convergence and mean squared convergence.
\begin{ln-define}{Uniform Convergence}{}
    A strong criterion of the series' validity (its correctness) is the uniform convergence of this limit: for all real number $\epsilon > 0$, there exists a positive integer $M$ such that for all $t \in \mathbb{R}$ and for all $N > M$,
    \[
        |x(t) - x_N(t)| < \epsilon
    \]
    A sufficient condition for such criterion is that the signal $x$ is continuous and has a piecewise continuous first derivative.
\end{ln-define}
\begin{ln-define}{Mean Square Convergence}{}
    Mean Square Convergence is a weaker condtiion for a Fourier Series' validity, where the total energy in the error over one peiod is zero.
    The concept of energy in error is not in scope, so understanding its mathematical representation is fine enough.
    So, a Fourier Series $x_N(t)$ converges in mean square to $x(t)$ if:
    \[
        \lim_{N \rightarrow \infty} \int_0^p {|x(t) - x_N (t)|}^2 \,dt = 0
    \]
    And, with some other mathematical facts, we may conclude that the following condition indicates mean square convergence:
    \[
        \lim_{N \rightarrow \infty} \int_0^p {|x(t)|}^2 \,dt = 0
    \]
\end{ln-define}

\section{Discrete-time Signals}
We mentioned previously that computers mostly work with discrete-time signals, which are signals in the form $x: \mathbb{Z} \rightarrow \mathbb{R}$.
Discrete-time signals can also be decomposed into sinusoidal components like continuous-time signals, with attention to the following subtleties:

\subsection{Periodicity}
A discrete-time signal is periodic if there exists a natural number $p$ such that:
\[
    \forall n \in \mathbb{Z}, x(n + p) = x(n)
\]
However, not all sinusoidal discrete-time signals are periodic.
Consider the signal:
\[
    x(n) = \cos(2\pi f n)
\]
In the continuous-time version, cosine waves have a period of $2 \pi$ (as well as the multiples of $2\pi$), so for the discrete time version of such signal to be periodic, we require a period $p$ such that $fp$ is a nonzero integer.
Alternatively, this is to say that for discrete timesteps $n$, when $n = p$, it is required that $fp$ is an integer such that the cosine signal is periodic as defined in its continuous-time version.
And therefore, we can state that the period of a discrete sinusoid is $p = \frac{K}{f}$ where $K$ is the smallest natural number such that $\frac{K}{f}$ is a nonzero integer.

\subsection{Discrete-Time Fourier Series}
A Fourier Series' prototype reamins to be:
\[
    x(n) = A_0 + \sum_{k = 1}^K A_k \cos(k \omega_0 n + \phi_k)
\]
with the remark that:
\[
    K = 
    \begin{cases}
        \frac{p - 1}{2} &\text{if $p$ is odd} \\
        \frac{p}{2} &\text{if $p$ is even}
    \end{cases}
\]
which brings us to an interesting point: discrete-time Fourier Series are never infinite, because discrete-time signals cannot represent frequenceis above a certain value based on the previously posted constraints on periodicity. This will be examined further in the last few chapters of this note.
