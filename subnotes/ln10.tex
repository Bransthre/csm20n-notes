\chapter{It's Time to Get a Little Complex}

\section{Review of Imaginary Numbers}
There are several things in this world we claim ``imaginary''; for example, a 4.0 GPA, a job that makes you fulfilled the entire life, a lifelong motivation, and my friends.
But, numbers may be imaginary too. This naming comes from the sense that there are ``real'' numbers we can somehow quantify, and ``imaginary'' numbers that violate the rules of mathematics.
Particularly, imaginary numbers are multiples of this value $\sqrt{-1}$:
\[
    ImaginaryNumbers = \{y \sqrt{-1} | y \in \mathbb{R}\}
\]
As an abbreviation, we shall from here on denote $i = \sqrt{-1}$.

The arithmetics of imaginary numbers can be easily summarized with the following rules:
\begin{enumerate}
    \item $\forall y_1, y_2 \in \R, i y_1 + i y_2 = i (y_1 + y_2)$
    \item $i ^ 2 = -1$
    \item $\frac{i}{i} = 1$
\end{enumerate}

\section{Review of Complex Numbers}
\subsection{Definition of Complex Numbers}
A complex number is the sum of a real number $x$ and an imaginary number $iy$:
\[
    \mathbb{C} = {x + y\sqrt{-1}| x, y \in \mathbb{R}}
\]
Alternatively, we can also express a complex number $z$ in its decomposed form:
\[
    Z = Re\{z\} + i Im\{z\}
\]
where $Re\{z\}$, $Im\{z\}$ are respectively the real component of the complex number and the coefficient of the imaginary component of complex number.

The fundamental arithmetics of complex numbers follows precisely those rules for imaginary numbers.
Outside of those rules, there are some interesting arithmetic properties.
The first is the definition of \textbf{complex conjugate}:
\begin{ln-define}{Complex Conjugate}{}
    The complex conjugate of a complex number $x + iy$ is defined to be $x - iy$.
    We denote the complex conjugate for a complex number $z = x + iy$ to be $z^* = x - iy$, such that:
    \begin{align*}
        Re\{z\} &= \frac{z + z^*}{2} \\
        Im\{z\} &= \frac{z - z^*}{2} \\
        z \times z^* &= x^2 + y^2
    \end{align*}
    We also call the positive square root of $z \times z^*$ the modulus (or magnitude) of $z$.
\end{ln-define}

\section{Trigonometric Representation of Complex Numbers}
\subsection{Exponentials}
Remember from Calculus 2 (or Calculus BC) that the exponential of a real number $x$ can be expressed as an infinite-term polynomial:
\[
    e^x = \sum_{k=0}^\infty \frac{x^k}{k}
\]
And such infinite series expansion exists for cosine and sine as well:
\begin{align*}
    \cos(\theta) &= \sum_{k=0}^\infty {(-1)}^k \frac{\theta^(2k)}{(2k)!} \\
    \sin(\theta) &= \sum_{k=0}^\infty {(-1)}^k \frac{\theta^(2k + 1)}{(2k + 1)!}
\end{align*}

And through some derivation, we can discover the following formula:
\begin{ln-theorem}{Euler's Formula}{}
    \textbf{\textit{Statement.}} We may claim that:
    \[
        e^{i \theta} = \cos(\theta) + i \sin(\theta)
    \]
    \tcblower
    \textbf{\textit{Proof.}} Using the previously exhibited infinite series expansions for exponentials and trigonometric functions:
    \begin{align*}
        e^{i \theta}
        &= 1 + (i \theta) + \frac{{(i \theta)}^2}{2!} + \frac{{(i \theta)}^3}{3!} + \cdots \\
        &= [1 - \frac{\theta^2}{2!} + \frac{\theta^4}{4!} - \cdots] + i [\theta - \frac{\theta^3}{3} + \frac{\theta^5}{5} - \cdots] \\
        &= \cos(\theta) + i \sin(\theta)
    \end{align*}
\end{ln-theorem}
And using such formula, we can even derive that:
\begin{align*}
    \cos(\theta) &= \frac{e^{i \theta} + e^{-i \theta}}{2} \\
    \sin(\theta) &= \frac{e^{i \theta} - e^{-i \theta}}{2i}
\end{align*}

% Can use this to make proof questions about trigonometric identities!

\subsection{Polar Coordinates}
Previously we have expressed complex numbers in its real-imaginary decomposition, such that $z = x + i y$.
This is known as a Cartesian representation.

An alternative method of expressing complex numbers is using polar coordinates, which conveys the following translation rules from a Cartesian representation:
\begin{align*}
    z &= x + i y \\
    &= |z| (\frac{x}{|z|} + i \frac{y}{|z|}) \\
    &= |z| (\cos(\theta) + i \sin(\theta)) = |z| e^{i \theta}
\end{align*}
where $\theta = \tan^{-1} (\frac{y}{x})$ is the angle between the vector representation of $z$, $\begin{bmatrix} x & y \end{bmatrix}^T$, and the positive x-axis.
Here, we then find that we can express any complex number in a two-dimensional coordinate system $(|z|, \theta)$.

Interestingly, following all arithmetic rules we have discussed before, we would find that
\[
    z_1 z_2 = |r_1||r_2| e^{i (\theta_1 + \theta_2)}
\]
From this, we make make two conclusions:
\begin{bindenum}
    \item $|z_1 z_2| = |z_1| |z_2|$, which entirely follows the polar representation of a complex number.
    \item $\angle (z_1, z_2) = \angle(z_1) + \angle(z_2)$, which we can see from the added exponents in the polar representation of product of complex numbers.
\end{bindenum}
