\chapter{Infinite State Machine}

\section{Definition of ISM}

\subsection{Mathematical Abstraction of ISM}
An infinite state machine is fundamentally a state machine that has infinite possible states.
These machines have just the same components as a usual state machine does, except the spaces $States$, $Inputs$, $Outputs$ now become:
\[
    \begin{cases}
        States &= \R^n \\
        Inputs &= \R^m \\
        Outputs &= \R^k
    \end{cases}
\]
for some arbitrary $n$, $m$, $k$.

For the sake of terminology, let us also discuss the dimensionality of such systems.
A system that receives one-dimensional inputs and provide one-dimensional outputs is known as a single-input, single-output (SISO) system.
On the opposite side is a multiple-input, multiple-output (MIMO) system, which is when $m > 1$ and $k > 1$.
This knowledge wouldn't really contribute a lot to your knowledge of the syllabus content.

\subsection{Composition of ISM}
Remember that an infinite state machine is still the 5-components tuple of:
\[
    M = (States, Inputs, Outputs, update, initialState)
\]
then, besides the above listed change, we should also observe that the $update$ function can now be defined as follows:
\[
    update: \R^n \times \R^m \rightarrow \R^n \times \R^k
\]
as it takes in an input signal, a current state, and outputs an output signal and a current state.

So, just as any system, an infinite state machine has the necessary components to build an update loop. That is, an equation that decides the next state of the system:
\[
    \forall n \in \mathbb{Z}, n \geq 0, s(n+1) = nextState(s(n), x(n))
\]
and an equation that determines the output system
\[
    \forall n \in \mathbb{Z}, n \geq 0, y(n) = output(s(n), x(n))
\]
The composition of the output loop builds what we call the ``state-space model'' of our system.

\subsection{Involvement of Time Measurement}
Note that the unit of time in our machine's state evolution is still ``step''.
Across each ``step'', some input is provided to the machine for a stepwide update.
And, because step is a discrete unit of time (you can only have a nonnegative integer amount of steps passed in a system), step becomes what we call the ``time index'' of the system, which is the identifier of time for in-machine events.
When the update rules ($nextState$ and $output$) of the system doesn't change across steps, we call it a \textbf{time-invariant} system.
